\documentclass[]{article}
\usepackage{amsmath}
\usepackage{amssymb}
\usepackage{verbatim} %con verbatim escribes bloques de texto con letra mono.
\usepackage{graphicx} %para insertar imagenes, cuando meta yo una usa el codigo de ejemplo
\usepackage{listings}
\usepackage{fullpage}
\usepackage{color}
\usepackage{fancyvrb}
\usepackage[spanish]{babel}
\usepackage[utf8]{inputenc} %Para usar acentos directamente en latex
\usepackage{hyperref} %Para que el indice tenga hiperenlaces y si quieres poner los tuyos
\hypersetup{%
	pdfborder = {0 0 0}
}

\definecolor{mygreen}{rgb}{0,0.6,0}
\definecolor{mygray}{rgb}{0.5,0.5,0.5}
\definecolor{mymauve}{rgb}{0.58,0,0.82}

%Para insertar código: crea un recuadro con texto mono y lineas enumeradas. Puedes referenciar un fichero y no copiar y pegar aquí.
\lstset{ %
	backgroundcolor=\color{white},   % chohttp://xdxd.com/ose the background color; you must add \usepackage{color} or \usepackage{xcolor}
	basicstyle=\footnotesize,        % the size of the fonts that are used for the code
	breakatwhitespace=false,         % sets if automatic breaks should only happen at whitespace
	breaklines=true,                 % sets automatic line breaking
	captionpos=b,                    % sets the caption-position to bottom
	commentstyle=\color{mygreen},    % comment style
	frame=single,                    % adds a frame around the code
	keepspaces=true,                 % keeps spaces in text, useful for keeping indentation of code (possibly needs columns=flexible)
	numbers=left,                    % where to put the line-numbers; possible values are (none, left, right)
	numbersep=5pt,                   % how far the line-numbers are from the code
	numberstyle=\tiny\color{mygray}, % the style that is used for the line-numbers
	rulecolor=\color{black},         % if not set, the frame-color may be changed on line-breaks within not-black text (e.g. comments (green here))
	showspaces=false,                % show spaces everywhere adding particular underscores; it overrides 'showstringspaces'
	showstringspaces=false,          % underline spaces within strings only
	showtabs=false,                  % show tabs within strings adding particular underscores
	stepnumber=1,                    % the step between two line-numbers. If it's 1, each line will be numbered
	stringstyle=\color{mymauve},     % string literal style
	tabsize=4,
	inputencoding=utf8,
	title=\lstname                   % show the filename of files included with \lstinputlisting; also try caption instead of title
}



\title{Seguridad}
\author{José Luis Cánovas Sánchez - 48636907A\\Ezequiel Santamaría Navarro - 20096517Z}

\begin{document}

\maketitle

\begin{abstract}
En esta memoria se describe el desarrollo del apartado 2.2 de las prácticas de Seguridad:
%TODO
\end{abstract}

\tableofcontents


\section{Introducción}

Como grupo 3 de prácticas, desplegaremos las organizaciones 31 y 32.
El trabajo en grupo ha consistido en que ambos hemos desarrollado y configurado todo conjuntamente,
lo que se refleja en nuestro repositorio git de GitHub, donde subimos los scripts de instalación
con los que automatizamos la práctica, pues preferimos no usar máquinas virtuales, por los problemas
 de fluidez y tamaño de las máquinas virtuales. Así podemos trabajar con unos pocos megas de scripts y capturas en cualquier ordenador.


%\begin{center}
%	\includegraphics[width=1\linewidth]{images/}
%\end{center}


\section{}

\subsection{Scripts de instalación}

Para automatizar la configuración del escenario en cualquier máquina, hemos escrito una serie de scripts que no necesitan de interacción manual una
vez se inicia la instalación (excepto algún \textit{Enter} ocasional). Utilizamos la herramienta \textit{make} para facilitar aún más
el despliegue. La organización de los scripts corresponde a una jerarquía de directorios como la siguiente:

\hfil

%TODO: modificar
\begin{BVerbatim}
ROUTER
	Makefile
	network
	test

ORG1
	SERVER
		Makefile
		network
		test
		ESCENARIO
			install.sh
			php-oauth-saml-demo-master
				appserver
				php-oauth
				simplesamlphp
				vagrant

ORG2
	SERVER
		Makefile
		network
		test
		ESCENARIO
			install.sh
			php-oauth-saml-demo-master
				appclient
				vagrant
\end{BVerbatim}

\hfill


Donde ESCENARIO sirve como carpeta contenedora de los scripts específicos de esta práctica, mientras que otros ficheros como \textit{network} nos
permiten configurar LEGO automáticamente, configurando la red, el forwarding, la resolución de DNS o fichero \textit{hosts}, etc.


\hfill

En la práctica, para ejecutarlos, basta con irse al directorio correspondiente, por ejemplo, para configurar la Organización 31, navegamos a ORG1/SERVER/, y desde ahí
ejecutamos la orden:

%TODO:
\begin{verbatim}
sudo make [network] [escenario] [test] [...]
\end{verbatim}

Si no ponemos argumentos, por defecto aplicaría la configuración de la red y test. De querer instalar la práctica 2, deberíamos indicar \textit{make saml}, y, para el escenario a entregar, \textit{make escenario}.

\hfill

%TODO:
Es importante notar que la configuración de la red (\textit{make network}) no sólo configura las IP de cada máquina, así como el \textit{forwarding} de la que haga de Router, sino que además configura el DNS de Google para navegar por internet, y en el fichero \textit{/etc/hosts} añade las entradas de \textit{server.org31}, \textit{server.org32}, \textit{oauth.org31}, \textit{appclient.org32}, etc., que se pueden ver en la imagen del escenario al inicio de esta sección. Esto último es de gran importancia porque así pasamos de usar los nombres \textit{*.local} del escenario de la práctica 3, a direcciones cualquiera que podrían estar en la jerarquía DNS, y al forzarnos a cambiarlos, nos daremos cuenta de cómo se podría mejorar este escenario de prácticas.




\end{document}
